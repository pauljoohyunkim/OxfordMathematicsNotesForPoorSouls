%        File: ComplexAnalysis.tex
%     Created: Sat Jun 24 03:00 PM 2023 B
% Last Change: Sat Jun 24 03:00 PM 2023 B
%
\documentclass[a4paper, 12pt]{article}
\usepackage[]{amsmath}
\usepackage{amsthm}
\usepackage{amssymb}
\usepackage{mathtools}

\newcommand{\R}{\mathbb{R}}
\newcommand{\C}{\mathbb{C}}

\title{Complex Analysis Summary}
\author{Paul Joo-Hyun Kim}
\begin{document}
\maketitle
\setcounter{section}{-1}
\section{Preface}
This note is for people studying complex analysis,
and got lost in the middle with bunch of technical explanations.
I wil try my best to be succinct as possible,
stating important results (mostly without proof, but a bit of justification).

\textbf{Warning}: This summary note is not a substitute for the lecture note.
Make sure you study from lecture note!

\section{Complex Plane and M\"obius Maps}
We will be working in what's known as the \textit{extended complex plane}.
Redefine the symbol $\C \coloneqq \C \cup \left\{ \underbrace{\infty}_{\text{Complex Infinity}} \right\}$;
that is, whenever I mention $\C$, I refer to the space of complex numbers
and infinity.

Note that in $\C$, $\infty$ is different from infinity in real numbers.
$\infty \coloneqq \frac{1}{0}$ is a value that is not ``larger'' or ``smaller'' than any number
(since we are talking about complex number\dots), but rather
a number on a complex plane at a really far distance from origin.

It is \textbf{WRONG} to say:
\begin{itemize}
    \item $\infty \geq a$ for any $a \in \C$
    \item $\infty \leq a$ for any $a \in \C$
\end{itemize}
However, it is \textbf{CORRECT}\footnote{
    Subtlety here: it seems a bit dodgy to say $\infty = \infty$,
    but this is matter of definition;
    you won't really encounter this type of ``philosophical'' problem
    in your exam.
} to say:
\begin{itemize}
    \item $|\infty| \geq |a|$ for any $a \in \C$.
\end{itemize}
\end{document}


